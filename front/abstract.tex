
% -------------------------------------------------------
%  Abstract
% -------------------------------------------------------


\شروع{وسط‌چین}
\مهم{چکیده}
\پایان{وسط‌چین}
\بدون‌تورفتگی
ترمیم تصاویر، به‌عنوان یکی از چالش‌های اصلی در پردازش تصویر، با هدف بازسازی قسمت‌های از‌دست‌رفته یا آسیب‌دیده‌ی تصاویر بدون تخریب ساختار بصری و مفهومی انجام می‌شود. روش‌های مرسوم ترمیم، شامل تکنیک‌های الگوریتمی و یادگیری‌محور، اغلب در شرایط پیچیده مانند بازیابی جزئیات دقیق یا بازسازی بافت‌های پیچیده محدودیت‌هایی دارند. در این پژوهش، یک مدل نوین مبتنی بر معماری Transformers پیشنهاد شده است که با بهره‌گیری از قدرت مکانیزم توجه، عملکرد قابل‌توجهی در بازیابی ساختار و جزئیات ارائه می‌دهد.


در این مطالعه، پس از مرور جامع بر روش‌های پیشین و ارزیابی محدودیت‌های آن‌ها، اصلاحاتی در معماری Transformer استاندارد اعمال شده و کارایی آن با استفاده از معیارهای استاندارد مانند SSIM و PSNR سنجیده شده است. همچنین، آزمایش‌های جانبی با استفاده از LSTM‌ ها بر روی داده‌های غیرزمانی انجام شده که اگرچه به نتیجه مطلوب نرسیدند، اما بینش‌های ارزشمندی در خصوص ویژگی‌های این داده‌ها ارائه کردند. 


\پرش‌بلند
\بدون‌تورفتگی \مهم{کلیدواژه‌ها}: 
ترمیم تصویر، ترنسفورمر، مکانیزم توجه، شبکه های عصبی
\صفحه‌جدید
