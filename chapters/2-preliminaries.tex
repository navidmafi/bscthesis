
\chapter{مفاهیم اولیه}

در این فصل،‌ ابتدا مسئله ترمیم تصویر را بطور مفصل تشریح و تعریف کرده و معماری‌های پایه را بررسی می‌کنیم. سپس به بررسی مجموعه‌داده‌های استفاده شده در یادگیری این روش ها می‌پردازیم.

\section{مسئله ترمیم تصویر}


ترمیم تصویر به معنای بازسازی نواحی گمشده یا خراب در یک تصویر است. این نواحی ممکن است به دلایل مختلفی مانند آسیب فیزیکی به تصویر، حذف بخش‌هایی از تصویر یا نقص داده‌ها در حین پردازش رخ دهند. مسئله ترمیم تصویر را می‌توان به‌طور ریاضی به صورت زیر بیان کرد:

فرض کنید که یک تصویر $I$ از ابعاد $H \times W$ داریم که بخشی از آن خراب یا گمشده است. تصویر خراب‌شده را می‌توان به صورت $I_{\text{observed}}$ نمایش داد که به‌طور جزئی از $I$ در نواحی مشخص شده توسط یک ماتریس ماسک $M$ به‌دست آمده است. به این صورت که:

$$
I_{\text{observed}} = M \odot I
$$

در این رابطه، $\odot$ نشان‌دهنده ضرب عنصر به عنصر است و $M$ یک ماتریس باینری است که در آن مقادیر 1 به نواحی دیده‌شده و مقادیر 0 به نواحی گمشده اشاره دارند.

هدف از ترمیم تصویر، بازسازی نواحی گمشده ($I_{\text{missing}}$) به‌طوری است که تصویر ترمیم‌شده $I_{\text{restored}}$ به‌صورت زیر حاصل شود:

$$
I_{\text{restored}} = I_{\text{observed}} + I_{\text{missing}}
$$

که در آن $I_{\text{missing}}$ نواحی گمشده است که باید توسط مدل ترمیم بازسازی شوند. هدف مدل ترمیم، یادگیری یک تابع $f_{\theta}$ است که بتواند $I_{\text{missing}}$ را بر اساس $I_{\text{observed}}$ بازسازی کند:

$$
I_{\text{missing}} = f_{\theta}(I_{\text{observed}})
$$

که در آن $f_{\theta}$ یک تابع است که در روش هایی که بر پایه یادگیری عمیق عمل می‌کنند، پارامترهای آن به‌وسیله داده‌های آموزشی بهینه می‌شوند.

