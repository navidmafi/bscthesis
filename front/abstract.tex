
% -------------------------------------------------------
%  Abstract
% -------------------------------------------------------


\شروع{وسط‌چین}
\مهم{چکیده}
\پایان{وسط‌چین}
\بدون‌تورفتگی

ترمیم تصاویر یکی از زمینه‌های مهم و چالش‌برانگیز در پردازش تصویر و بینایی کامپیوتر است که کاربردهای گسترده‌ای از جمله بازیابی تصاویر تاریخی، ویرایش تصاویر دیجیتال و حذف اشیای ناخواسته دارد. روش‌های سنتی ترمیم تصویر، اگرچه ساده و کم‌هزینه بودند، اما در تولید نتایج با کیفیت بالا به‌ویژه در شرایط پیچیده ناکارآمد ظاهر شدند. ظهور شبکه‌های عصبی و معماری‌های مبتنی بر یادگیری عمیق مانند شبکه‌های بازگشتی (RNN) و شبکه‌های کاهشی-افزایشی (Encoder-Decoder) تحولی در این زمینه ایجاد کرد، اما محدودیت‌هایی نظیر وابستگی طولانی‌مدت و توانایی پردازش داده‌های غیرخطی همچنان وجود داشت.
در این پژوهش، با معرفی و بررسی معماری‌های ترانسفورمر، که در ابتدا برای وظایف ترجمه ماشینی طراحی شده بودند، به کاربردهای آن‌ها در ترمیم تصاویر می‌پردازیم. مدل پیشنهادی ما با بهره‌گیری از ویژگی‌های ذاتی ترانسفورمرها در پردازش وابستگی‌های بلندمدت و حفظ انسجام مکانی-زمانی، عملکرد قابل توجهی در بازسازی تصاویر آسیب‌دیده از خود نشان می‌دهد.
نتایج تجربی حاصل از ارزیابی مدل بر اساس معیارهای استاندارد مانند SSIM و PSNR نشان‌دهنده بهبود چشمگیر دقت و کیفیت در مقایسه با روش‌های پیشین است. همچنین، آزمایشات جانبی ما در استفاده از شبکه‌های بازگشتی برای وظایف غیرزمانی، علیرغم شکست، بینش‌های ارزشمندی را فراهم کرد که در طراحی مدل نهایی مورد استفاده قرار گرفتند.
این پژوهش نشان‌دهنده پتانسیل بالای معماری‌های ترانسفورمر در رفع محدودیت‌های روش‌های قبلی و گامی مؤثر در جهت توسعه روش‌های نوین ترمیم تصاویر است.

\پرش‌بلند
\بدون‌تورفتگی \مهم{کلیدواژه‌ها}:
ترمیم تصویر، ترنسفورمر، مکانیزم توجه، شبکه های عصبی
\صفحه‌جدید
